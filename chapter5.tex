\chapter{جمع بندی و نتیجه گیری و  کارهای آینده}\label{chapter5}

\section{جمع بندی و نتیجه گیری}

در این پروژه به طراحی و پیاده‌سازی یک بستر بدون سرور اینترنت اشیاء پرداخته شد. این کار پس از نیازسنجی در زمینه‌های اینترنت اشیاء و سرویس‌های ابری صورت گرفت. همچنین راه حل‌های مختلف و تکنولوژی‌های موجود مورد ارزیابی قرار گرفت و راه حل نهایی انتخاب و بررسی شد. در آخر طراحی یک سرویس گیرنده برای آزمایش کردن بستر توضیح داده شد.

همانطور که در فصل‌های قبل گفته شد، با پیشرفت‌های اخیر اینترنت اشیاء، استفاده از سرویس‌های ابری امری ضروری شده است. این سرویس‌ها می‌توانند در سطح‌های مختلفی ارائه شوند و هرکدام برای کاربردهای مختلف مزایا و معایبی دارند. با طراحی و استفاده از بستر بدون سرور اینترنت اشیاء نتیجه گرفته شد که استفاده از این بستر مزایا و برتری‌های زیر را برای توسعه‌دهندگان دارد: 

\begin{enumerate}
	
	\item سرعت بخشیدن به توسعه نرم‌افزاری و سخت‌افزاری و کاهش هزینه‌ها با حذف نیاز به مدیریت زیرساخت، مناسب شرکت‌های نوپا و استارت آپ‌ها
	
	\item توانایی ارائه‌ سرویس در سطح نرم‌افزار برای تیم‌های سخت‌افزاری بدون نیاز به داشتن دانش در حوزه نرم‌افزار
	
	\item قابلیت مقیاس‌پذیری و تعادل بار با توجه به استفاده از تکنولوژی‌های جدید و معماری میکروسرویس
	
	\item پیاده‌سازی منطق برنامه کاربردی در سطح تابع با هر زبان برنامه‌نویسی
	
\end{enumerate}

\newpage

\section{کارهای آینده}

کارهای انجام شده در این پروژه مقدمه‌ای بود بر طراحی یک بستر جامع اینترنت اشیاء و بدون سرور. همانطور که در فصل سوم گفته شد، بستر اینترنت اشیاء می‌تواند دارای قابلیت‌های بسیار زیادی باشد که در این پروژه تنها به یک قابلیت آن پرداخته شد. بسیاری از این قابلیت‌ها  همچنین ترکیب این قابلیت‌ها با بستر بدون سرور امکان‌های بیشتری به توسعه دهندگان می‌دهد. در ادامه، کارهای آینده برای توسعه بیشتر و بهبود این بستر پیشنهاد می‌شوند:

\begin{enumerate}

	\item اضافه نمودن قابلیت کنترل دستگاه‌های اینترنت اشیاء با ارسال دستور به آن‌ها
	
	\item استفاده از \lr{package} داخلی بستر \lr{OpenWhisk} به نام \lr{alarms} برای ایجاد رویدادها بر اساس زمان
	
	\item استفاده از \lr{package} های داخلی دیگر مانند \lr{websocket}، \lr{pushnotifications} و ... برای بهبود قابلیت‌های بستر

	\item پیاده‌سازی سیستم‌های \lr{logging} و \lr{monitoring} برای بهبود روند توسعه‌ی برنامه‌های کاربردی

	\item فعال‌سازی قابلیت مقیاس پذیری خودکار و پیاده‌سازی در بستر‌های ارائه‌دهندگان ابری
	
\end{enumerate}

