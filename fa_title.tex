%% -!TEX root = AUTthesis.tex
% در این فایل، عنوان پایان‌نامه، مشخصات خود، متن تقدیمی‌، ستایش، سپاس‌گزاری و چکیده پایان‌نامه را به فارسی، وارد کنید.
% توجه داشته باشید که جدول حاوی مشخصات پروژه/پایان‌نامه/رساله و همچنین، مشخصات داخل آن، به طور خودکار، درج می‌شود.
%%%%%%%%%%%%%%%%%%%%%%%%%%%%%%%%%%%%
% دانشکده، آموزشکده و یا پژوهشکده  خود را وارد کنید
\faculty{دانشکده برق}
% گرایش و گروه آموزشی خود را وارد کنید
\department{گرایش الکترونیک}
% عنوان پایان‌نامه را وارد کنید
\fatitle{طراحی و پياده‌سازی بستر بدون سرور اینترنت اشیاء}
% نام استاد(ان) راهنما را وارد کنید
\firstsupervisor{دکتر طاهری}
%\secondsupervisor{استاد راهنمای دوم}
% نام استاد(دان) مشاور را وارد کنید. چنانچه استاد مشاور ندارید، دستور پایین را غیرفعال کنید.
%\firstadvisor{نام کامل استاد مشاور}
%\secondadvisor{استاد مشاور دوم}
% نام نویسنده را وارد کنید
\name{محمدحسن }
% نام خانوادگی نویسنده را وارد کنید
\surname{مجاب}
%%%%%%%%%%%%%%%%%%%%%%%%%%%%%%%%%%
\thesisdate{شهریور 98}

% چکیده پایان‌نامه را وارد کنید
\fa-abstract{
\fontsize{12pt}{18pt}\selectfont
با توجه به گسترش اینترنت اشیاء در زمینه های تحقیقاتی و کاربردی احساس نیاز به سرویس‌های ابری اینترنت اشیاء بیشتر شده است.  بیشتر شرکت‌ها و استارت آپ‌هایی که در این حوزه فعال هستند نیز تنها در سطح سخت‌افزار و توسعه نرم‌افزار فعالیت می‌کنند و برای ذخیره‌سازی داده‌ها و پردازش‌ها از ارائه دهنده‌های ابری استفاده می‌کنند.
همزمان با اینترنت اشیاء، معماری بدون سرور نیز در زمینه رایانش ابری در حال رشد است که تمامی هزینه و تقاضای سرورها به عهده‌ی ارائه دهنده‌ی سرویس است و تنها مدت زمان اجرای توابع معیار قرار می‌گیرد. همچنین توسعه دهنده‌ها نیازی به در نظر گرفتن شرايط زيرساخت و سرورها ندارند. هدف این پروژه راه اندازی یک بستر بدون سرور برای اجرای توابع و پردازش های شبکه های اینترنت اشیاء شامل دریافت و تحلیل داده‌های حسگرها و اجرای سناریوهای مشخص است. این بستر می‌تواند مورد استفاده استارت آپ‌ها، دانشگاه‌ها و مرکزهای تحقیقاتی اینترنت اشیاء قرار‌ گیرد.
}


% کلمات کلیدی پایان‌نامه را وارد کنید
\keywords{اینترنت اشیاء، بدون سرور، تابع به عنوان سرویس}



\AUTtitle
%%%%%%%%%%%%%%%%%%%%%%%%%%%%%%%%%%