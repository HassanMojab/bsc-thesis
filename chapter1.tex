\chapter{معرفی}

\section{مقدمه‌}

وب جهان‌گستر\LTRfootnote{Wolrd Wide Web} طی ۲۰ سال گذشته بیش از ۷ میلیارد کاربر داشته و سرعت رشد آن بیش از ۱۰۰۰ درصد افزایش یافته است. این تکنولوژی در سیر تکامل سایر تکنولوژی هایی که روزانه با آن‌ها سر و کار داریم، تحول اساسی ایجاد کرده است. از تلفن‌های هوشمند و ساعت‌های هوشمند گرفته تا هواشناسی، کشاورزی، خانه‌های هوشمند و ماشین‌های هوشمند همگی تحت تأثیر این تحول قرار گرفته‌اند، به صورتی که می‌توان گفت اینترنت، شیوه زندگی کردن انسان‌ها را عوض کرده ‌است.

علاوه‌ بر کاربران عادی، دسترسی به اینترنت برای اکثر دستگاه‌ها و وسایل ممکن است. با پیشرفت‌های اخیر در زمینه الکترونیک دیجیتال و ارتباط بی‌سیم می‌توان گفت برای تمامی اشیاء این قابلیت به وجود آمده است که با اینترنت به راحتی به یکدیگر متصل شوند، با هم ارتباط داشته باشند و توسط کاربران یا اشیاء دیگر نظارت یا حتی کنترل شوند. این مفهوم به اینترنت اشیاء\LTRfootnote{Internet of Things} معروف شده است و در دهه اخیر رشد چشمگیری داشته است. در حال حاضر تعداد دستگاه‌هایی که به اینترنت متصل هستند از تعداد انسان‌ها پیشی گرفته است و پیش‌بینی می‌شود تا ۱۰ سال آینده به ۵۰۰ میلیارد برسد. دستگاه‌های اینترنت اشیاء در معماری، میزان حافظه و توان مصرفی با یکدیگر تفاوت دارند. شبکه‌های اینترنت اشیاء نیز همانند دستگاه‌ها و بسته به محیط متنوع هستند. همچنین این دستگاه‌ها در منابع محدودیت دارند که این محدودیت می‌تواند شامل حافظه، منبع انرژی یا توان پردازشی باشد.

دسترسی به اینترنت همچنین باعث تغییر شگرفی در حوزه خدمات فناوری اطلاعات شده است. از جمله این تغییرات می‌توان به پدیدار شدن رایانش ابری\LTRfootnote{cloud computing} و افزایش محبوبیت آن اشاره کرد. رایانش ابری مدلی از ارائه سرویس است که در آن ارائه دهندگان ابری، سرویس‌های پردازشی از جمله فضای ذخیره سازی، سرور‌ رایانشی و پایگاه داده‌ را از طریق اینترنت برای کاربر‌ها فراهم می‌کنند. هزینه متناسب با استفاده کاربران از این سرویس‌ها حساب می‌شود. رایانش ابری نه تنها باعث پایین آمدن هزینه‌های ساخت برنامه‌های کاربردی شده است، بلکه زمان دسترسی کاربران به زیرساخت‌ها و سرویس‌ها را کاهش داده است. در نتیجه بسیاری از شرکت‌ها به جای مراکز داده خصوصی خودشان به استفاده از سرویس‌های ابری روی آورده‌اند \cite{Kumar_2018}.
\newpage

\section{تعریف مسأله}

با توجه به گسترش اینترنت اشیاء در زمینه‌های تحقیقاتی و کاربردی و همچنین محدودیت‌های دستگاه‌های ‌‌اینترنت اشیاء، احساس نیاز به سرویس های ابری بیشتر شده است. رایانش ابری، منابع مورد نیاز شبکه‌های اینترنت اشیاء را فراهم می‌کند. بنابراین بیشتر شرکت‌ها و استارت آپ‌هایی که در این حوزه هستند تنها در سطح سخت‌افزار و توسعه نرم‌افزار فعالیت می‌کنند و برای ذخیره سازی داده‌ها و پردازش‌ها از ارائه دهنده‌های ابری استفاده می‌کنند. شركت‌های بزرگي همچون آمازون و گوگل كه در زمينه رایانش ابری پيشرو هستند، سرويس‌های متنوعي در زمينه زیرساخت به عنوان سرویس تا نرم‌افزار به عنوان سرویس ارائه مي‌كنند. این شرکت‌ها نمونه‌ای از ارائه دهندگان سرویس‌های ابری هستند که کاربران زیادی در سرتاسر جهان از سرویس‌هایشان استفاده می‌کنند. متأسفانه امكان دسترسي به اين سرويس‌ها توسط شرکت‌های ايراني يا اساساً وجود ندارد يا با هزينه‌های گزاف امكان پذير است. در داخل كشور نيز روند ارائه چنين سرويس‌هايي در حال آغاز است. بنابراين در بازار داخل عرضه بسيار كمتر از تقاضا است، بنابراين نياز به ارائه سرويس‌های نوین ابری به شدت احساس می‌شود.

استفاده از سرویس‌های ابری برای تأمین منابع دستگاه‌های اینترنت اشیاء در سطح‌های مختلف (زیرساخت، بستر نرم‌افزاری و نرم‌افزار) امکان‌پذیر است که هر کدام مزیت‌‌ها و معایب خاص خود را دارند. باید با بررسی دقیق‌تر و نیازسنجی، سرویس مناسب برای این حوزه مشخص گردد.

\section{اهداف}

همزمان با اینترنت اشیاء، معماری بدون سرور\LTRfootnote{serverless} نیز در زمینه رایانش ابری در حال رشد است. در این معماری هزینه و تقاضای سرورها به عهده‌ی ارائه دهنده سرویس است و تنها مدت زمان اجرای توابع تعریف شده توسط کاربر معیار قرار می‌گیرد. همچنین توسعه دهنده‌ها نیازی به در نظر گرفتن شرايط زيرساخت و سرورها ندارند و بدون نیاز به داشتن هیچ گونه اطلاعات فنی در این حوزه‌ به توسعه‌ی برنامه کاربردی خود با استفاده از تابع‌ها می‌پردازند.

هدف این پروژه طراحی، راه اندازی و بررسی مزایا و معایب یک بستر بدون سرور برای اجرای توابع و پردازش‌های شبکه‌های اینترنت اشیاء شامل دریافت و تحلیل داده های حسگرها و اجرای سناریوهای مشخص است. این بستر می‌تواند مورد استفاده استارت آپ‌ها، دانشگاه‌ها و مرکزهای تحقیقاتی اینترنت اشیاء قرار گیرد.

\section{ساختار}

این پایان‌نامه به جز فصل معرفی از چهار فصل دیگر تشکیل شده است. فصل \hyperref[chapter2]{دوم} به تعریف مفاهیم اولیه مورد نیاز و بررسی تکنولوژی‌ها و راه حل‌های موجود ‌پرداخته است. پس از یادگیری مفاهیم و در نظر گرفتن اهداف پروژه، در فصل \hyperref[chapter3]{سوم}، ابتدا ساختار کلی و طراحی پروژه توضیح داده شده، سپس نحوه پیاده سازی آن گام به گام مورد بررسی قرار گرفته است. در فصل \hyperref[chapter4]{چهارم} نحوه‌ی کار با پروژه با ایجاد یک برنامه‌ی کاربردی تست بررسی شده و نتایج ارزیابی‌های انجام گرفته ارائه می‌شود. نهایتاً در فصل \hyperref[chapter5]{پنجم}، کار‌های انجام شده در پروژه جمع‌بندی و نتیجه گیری می‌شود، سپس کارهای احتمالی آینده توضیح داده خواهد شد.

